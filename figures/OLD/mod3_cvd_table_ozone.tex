% latex table generated in R 3.6.1 by xtable 1.8-4 package
% Sat Apr  4 16:40:46 2020
\begin{tabular}{llp{6.5cm}rrrr}
  \hline
Model & Group & CCS codes & $n_{out}$ & $n_{obs}$ & \% change in rate below 25 \mu g \cdot m^{-3}$ (95\%CI) & \% change in rate above 25 \mu g \cdot m^{-3}$ (95\%CI) \\ 
  \hline
 1 &  1 & Other aneurysm (7.4.2.2) &  1 & 99,464 & -2.9 (-4.4, -1.3) & -2.9 (-4.4, -1.3) \\ 
   1 &  2 & Intracranial hemorrhage (7.3.1.1) &  1 & 413,040 & -2.4 (-3.2, -1.6) & -2.4 (-3.2, -1.6) \\ 
   1 &  3 & Transient cerebral ischemia (7.3.4) &  1 & 753,545 & -0.6 (-1.2, -0.1) & -0.6 (-1.2, -0.1) \\ 
   1 &  4 & Hypertension (7.1); heart valve disorders (7.2.1); peri-; endo-; and myocarditis; cardiomyopathy (except that caused by TB or STD) (7.2.2); acute myocardial infarction (7.2.3); coronary atherosclerosis and other heart disease (7.2.4); nonspecific chest pain (7.2.5); pulmonary heart disease (7.2.6); other and ill-defined heart disease (7.2.7); conduction disorders (7.2.8); cardiac dysrhythmias (7.2.9); cardiac arrest and ventricular fibrillation (7.2.10); occlusion of cerebral arteries (7.3.1.2); acute but ill-defined cerebrovascular accident (7.3.1.3); occlusion or stenosis of precerebral arteries (7.3.2); other and ill-defined cerebrovascular disease (7.3.3); late effects of cerebrovascular disease (7.3.5); peripheral and visceral atherosclerosis (7.4.1); aortic; peripheral; and visceral artery aneurysms (7.4.2); abdominal aortic aneurysm; without rupture (7.4.2.1); aortic and peripheral arterial embolism or thrombosis (7.4.3); other circulatory disease (7.4.4); diseases of veins and lymphatics (7.5) & 51 & 12,402,998 & 0.6 (0.4, 0.7) & 0.6 (0.4, 0.7) \\ 
   1 &  5 & Congestive heart failure; nonhypertensive (7.2.11) &  3 & 2,338,246 & 1.6 (1.2, 2.0) & 1.6 (1.2, 2.0) \\ 
   2 &  1 & Intracranial hemorrhage (7.3.1.1) &  1 & 413,040 & -3.1 (-4.1, -2.0) & -2.1 (-2.9, -1.2) \\ 
   2 &  2 & Other aneurysm (7.4.2.2) &  1 & 99,464 & -2.4 (-4.4, -0.4) & -2.8 (-4.5, -1.1) \\ 
   2 &  3 & Transient cerebral ischemia (7.3.4) &  1 & 753,545 & 0.2 (-0.5, 0.9) & -0.7 (-1.2, -0.1) \\ 
   2 &  4 & Hypertension (7.1); heart valve disorders (7.2.1); peri-; endo-; and myocarditis; cardiomyopathy (except that caused by TB or STD) (7.2.2); acute myocardial infarction (7.2.3); coronary atherosclerosis and other heart disease (7.2.4); nonspecific chest pain (7.2.5); pulmonary heart disease (7.2.6); other and ill-defined heart disease (7.2.7); conduction disorders (7.2.8); cardiac dysrhythmias (7.2.9); cardiac arrest and ventricular fibrillation (7.2.10); occlusion of cerebral arteries (7.3.1.2); acute but ill-defined cerebrovascular accident (7.3.1.3); occlusion or stenosis of precerebral arteries (7.3.2); other and ill-defined cerebrovascular disease (7.3.3); late effects of cerebrovascular disease (7.3.5); peripheral and visceral atherosclerosis (7.4.1); aortic; peripheral; and visceral artery aneurysms (7.4.2); abdominal aortic aneurysm; without rupture (7.4.2.1); aortic and peripheral arterial embolism or thrombosis (7.4.3); other circulatory disease (7.4.4); diseases of veins and lymphatics (7.5) & 51 & 12,402,998 & 0.9 (0.7, 1.1) & 0.4 (0.3, 0.5) \\ 
   2 &  5 & Congestive heart failure; nonhypertensive (7.2.11) &  3 & 2,338,246 & 1.7 (1.2, 2.1) & 1.5 (1.1, 1.9) \\ 
   \hline
\end{tabular}

