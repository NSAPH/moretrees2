% latex table generated in R 3.6.1 by xtable 1.8-4 package
% Sun Apr  5 18:46:35 2020
\begin{tabular}{llp{6.5cm}rrrr}
  \hline
Model & Group & CCS codes & $n_{out}$ & $n_{obs}$ & \% change in rate below 25 \mu g \cdot m^{-3}$ (95\%CI) & \% change in rate above 25 \mu g \cdot m^{-3}$ (95\%CI) \\ 
  \hline
 1 &  1 & Other aneurysm (7.4.2.2) &  1 & 99,464 & -3.3 (-4.8, -1.7) & -3.3 (-4.8, -1.7) \\ 
   1 &  2 & Intracranial hemorrhage (7.3.1.1) &  1 & 413,040 & -2.4 (-3.2, -1.6) & -2.4 (-3.2, -1.6) \\ 
   1 &  3 & Transient cerebral ischemia (7.3.4) &  1 & 753,545 & -0.4 (-1.0, 0.1) & -0.4 (-1.0, 0.1) \\ 
   1 &  4 & Nonspecific chest pain (7.2.5) &  1 & 1,191,085 & 0.0 (-0.4, 0.5) & 0.0 (-0.4, 0.5) \\ 
   1 &  5 & Hypertension (7.1); heart valve disorders (7.2.1); peri-; endo-; and myocarditis; cardiomyopathy (except that caused by TB or STD) (7.2.2); acute myocardial infarction (7.2.3); coronary atherosclerosis and other heart disease (7.2.4); pulmonary heart disease (7.2.6); other and ill-defined heart disease (7.2.7); conduction disorders (7.2.8); cardiac dysrhythmias (7.2.9); cardiac arrest and ventricular fibrillation (7.2.10); occlusion of cerebral arteries (7.3.1.2); acute but ill-defined cerebrovascular accident (7.3.1.3); occlusion or stenosis of precerebral arteries (7.3.2); other and ill-defined cerebrovascular disease (7.3.3); late effects of cerebrovascular disease (7.3.5); peripheral and visceral atherosclerosis (7.4.1); aortic; peripheral; and visceral artery aneurysms (7.4.2); abdominal aortic aneurysm; without rupture (7.4.2.1); aortic and peripheral arterial embolism or thrombosis (7.4.3); other circulatory disease (7.4.4); diseases of veins and lymphatics (7.5) & 50 & 11,211,913 & 0.5 (0.4, 0.7) & 0.5 (0.4, 0.7) \\ 
   1 &  6 & Congestive heart failure; nonhypertensive (7.2.11) &  3 & 2,338,246 & 1.4 (1.1, 1.8) & 1.4 (1.1, 1.8) \\ 
   2 &  1 & Intracranial hemorrhage (7.3.1.1) &  1 & 413,040 & -3.0 (-4.0, -2.0) & -2.1 (-3.0, -1.2) \\ 
   2 &  2 & Other aneurysm (7.4.2.2) &  1 & 99,464 & -2.7 (-4.7, -0.7) & -3.2 (-4.8, -1.5) \\ 
   2 &  3 & Transient cerebral ischemia (7.3.4) &  1 & 753,545 & 0.4 (-0.3, 1.0) & -0.5 (-1.1, 0.0) \\ 
   2 &  4 & Hypertension (7.1); heart valve disorders (7.2.1); peri-; endo-; and myocarditis; cardiomyopathy (except that caused by TB or STD) (7.2.2); acute myocardial infarction (7.2.3); coronary atherosclerosis and other heart disease (7.2.4); nonspecific chest pain (7.2.5); pulmonary heart disease (7.2.6); other and ill-defined heart disease (7.2.7); conduction disorders (7.2.8); cardiac dysrhythmias (7.2.9); cardiac arrest and ventricular fibrillation (7.2.10); occlusion of cerebral arteries (7.3.1.2); acute but ill-defined cerebrovascular accident (7.3.1.3); occlusion or stenosis of precerebral arteries (7.3.2); other and ill-defined cerebrovascular disease (7.3.3); late effects of cerebrovascular disease (7.3.5); peripheral and visceral atherosclerosis (7.4.1); aortic; peripheral; and visceral artery aneurysms (7.4.2); abdominal aortic aneurysm; without rupture (7.4.2.1); aortic and peripheral arterial embolism or thrombosis (7.4.3); other circulatory disease (7.4.4); diseases of veins and lymphatics (7.5) & 51 & 12,402,998 & 0.9 (0.7, 1.1) & 0.4 (0.2, 0.5) \\ 
   2 &  5 & Congestive heart failure; nonhypertensive (7.2.11) &  3 & 2,338,246 & 1.5 (1.1, 2.0) & 1.3 (1.0, 1.7) \\ 
   \hline
\end{tabular}

